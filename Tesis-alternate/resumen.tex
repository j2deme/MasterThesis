En el mundo actual existen una gran cantidad de aplicaciones enfocadas a la generación de información misma que bombardea a usuarios con datos no necesariamente relevantes, pues dichas aplicaciones presentan puntos de opinión sin tener fundamentos sustentables como investigaciones o referencias bibliográficas de índole científica o tecnológica. 

Esta problemática se presenta en diferentes nichos, por ejemplo, en el caso del sector salud el uso de imágenes de laboratorio facilita el diagnóstico temprano de malformaciones congénitas. Sin embargo, la interpretación de la información no es una tarea sencilla, pues se requiere realizar una serie de pasos para llegar a un diagnóstico satisfactorio. Otro ejemplo es el caso del análisis de los patrones de expresión en el rostro y cuerpo, empleados durante los test psicométricos o en evaluaciones psiquiátricas. Si bien es cierto que éstos pueden ser valorados por un especialista en el área, dichas valoraciones carecen de una métrica general independiente a la perspectiva del especialista. 

En ambos casos los problemas mencionados, actualmente se poseen metodologías y técnicas que facilitan la extracción de la información; sin embargo, no así el análisis de los datos proporcionados. Es en este punto donde los modelos computacionales juegan un papel relevante pues permiten establecer patrones y relaciones de la información obtenida. De forma más específica, los modelos computacionales enfocados a la web facilitan el acceso y uso de dichas herramientas de análisis a un gran número de personas distribuidas en diferentes puntos geográficos. 
