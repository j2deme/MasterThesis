% arara: pdflatex
% arara: bibtex
% arara: pdflatex
% arara: pdflatex
% arara: clean: { files: [Protocolo.aux, Protocolo.idx, Protocolo.ilg, Protocolo.ind, Protocolo.log]}
% arara: clean: { files: [Protocolo.bbl, Protocolo.blg, Protocolo.out, Protocolo.toc]}
\documentclass[english,ENG,MSc]{cinvestav} %si quieren en espanol, usen spanish, ESP
\usepackage{cite}
\usepackage{slashbox}
\usepackage[T1]{fontenc}
\usepackage[utf8]{inputenc}
\setcounter{secnumdepth}{3}
\setcounter{tocdepth}{3}
\usepackage{graphics,epsfig,latexsym,amssymb}
%\usepackage{graphics,epsfig,latexsym,amssymb}
\usepackage{tabularx}
\usepackage{titlesec}
\usepackage{amsmath}
%\usepackage[hypertex]{hyperref}
\usepackage{array}
\usepackage{float}
\usepackage{rotfloat}
\usepackage{textcomp}
\usepackage{graphicx}
\usepackage{color}
\usepackage{xcolor}
\usepackage{setspace}
\usepackage[TABBOTCAP]{subfigure}
\usepackage{algorithm}
\usepackage{captcont}
\usepackage{afterpage}
\usepackage{lscape}
\usepackage{algorithmic}
\usepackage{longtable}
\usepackage{caption}
\usepackage[authoryear,square]{natbib}

%\renewcommand{\thesubfigure}{\thefigure.\arabic{subfigure}} 
%\makeatletter 
%\renewcommand{\p@subfigure}{} 
%\renewcommand{\@thesubfigure}{\thesubfigure:\hskip\subfiglabelskip} 
%\makeatother 

%\usepackage[nonamebreak]{natbib}


\makeatletter

%%%%%%%%%%%%%%%%%%%%%%%%%%%%%% LyX specific LaTeX commands.
%% Because html converters don't know tabularnewline
\providecommand{\tabularnewline}{\\}
%% A simple dot to overcome graphicx limitations
\newcommand{\lyxdot}{.}

\floatstyle{ruled}
\newfloat{algorithm}{tbp}{loa}
\floatname{algorithm}{Algorithm}

%%%%%%%%%%%%%%%%%%%%%%%%%%%%%% User specified LaTeX commands.

%\usepackage{apacite}
%\usepackage{natbib}
%\let\citep=\citeA
%\let\cite=\citeA
\bibpunct{[}{]}{,}{n}{,}{,} 

%\usepackage[lined,boxed]{algorithm2e}
%\decimalpoint 

\title     {Un Estudio sobre los Mecanismos de dos Algoritmos Evolutivos Multi-objetivo}
\titleen        {A Study on the Mechanisms of two Multi-objective Evolutionary Algorithms}
\author         {Jorge Sebastian Hern\'{a}ndez Dom\'{i}nguez}
\department     {Laboratorio de Tecnolog\'{i}as de Informaci\'{o}n,
CINVESTAV-Tamaulipas}
\departmenten   {Information Technology Laboratory, CINVESTAV-Tamaulipas}
\degreein       {Ciencias en Computaci\'{o}n}
\degreeinen     {Computer Science}
\city           {Cd. Victoria, Tamaulipas, M\'{e}xico.}
\date           {\today}
\degreeday      {8}
\degreeyear     {2011}
\degreemonth  {Diciembre}
\degreemonthen  {December}
\acknowledgmenttoproject{
This research was partially funded by project number 51623 from ``Fondo Mixto
Conacyt-Gobierno del Estado de Tamaulipas''
}

%%%%%%%%%%%%%%%%%%%%%%%%%%%%%%%%%%%%%%%%%%%%%%%%%%%%%%%%%%%%%%%%%%%%%%%%

\chair          {Dr. Gregorio Toscano-Pulido}
\member         {Dr. Ricardo Landa Becerra}
\member         {Dr. Luis Gerardo de la Fraga}

%%%%%%%%%%%%%%%%%%%%%%%%%%%%%%%%%%%%%%%%%%%%%%%%%%%%%%%%%%%%%%%%%%%%%%%%

\dedication     {A Naruto}

%%%%%%%%%%%%%%%%%%%%%%%%%%%%%%%%%%%%%%%%%%%%%%%%%%%%%%%%%%%%%%%%%%%%%%%%




\abstract{
\vspace*{-7mm}
La \emph{Optimización Mediante Cúmulos de Partículas (PSO)} y la \emph{Evolución Diferencial (DE)} son dos \emph{Algoritmos Evolutivos (EAs)} simples de conceptualizar que han sido exitosamente utilizados para resolver problemas mono-objetivo. Dicha simplicidad y éxito han promovido su uso en problemas multi-objetivo. Aun cuando a la fecha existen varias propuestas \emph{PSO Multi-objetivo (MOPSO)} y \emph{DE Multi-objetivo (MODE)}, el conocimiento sobre el proceso de búsqueda que realizan estas dos metaheurísticas es escaso dado que solo existen algunos trabajos teóricos y únicamente mono-objetivo. Como resultado, no se conoce claramente el comportamiento de estos algoritmos evolutivos en problemas multi-objetivo. Esta tesis presenta un estudio empírico sobre estos dos \emph{Algoritmos Evolutivos Multi-objetivo (MOEAs)}. Dicho estudio consta de una serie de experimentos que comparan diferentes variantes en DE y fórmulas de vuelo en PSO. Después, se evalúa la manera en que estos dos MOEAs generan nuevas soluciones y se identifican características de dichas soluciones y su relación con los mecanismos presentes en ambos enfoques. Estos experimentos permitieron concluir que MOPSO se mueve agresivamente hacia regiones prometedoras lo cual puede deteriorar la búsqueda. MODE por otro lado, realiza una búsqueda pasiva basada en pasos pequeños que a la larga le permiten seguir moviéndose hacia el frente de Pareto. El conocimiento obtenido fue usado para diseñar dos nuevos MOEAs que mostraron ser competitivos al ser comparados con tres algoritmos (OMOPSO, NSGA-II y DEMO) representativos del estado del arte. 
}


\abstracten{
\vspace*{-7mm}
\emph{Particle Swarm Optimization (PSO)} and \emph{Differential Evolution (DE)} are two \emph{Evolutionary Algorithms (EAs)} which are very simple to conceptualize and have shown excellent results on single-objective optimization problems. As expected, this simplicity and success have promoted their migration to multi-objective optimization. Even when several \emph{Multi-Objective Particle Swarm Optimizers (MOPSOs)} and \emph{Multi-objective Differential Evolution (MODE)} algorithms are available to this date, knowledge about the search performed by these two metaheuristics is limited in regards to multi-objective optimization since only some theoretical single-objective studies have been developed. As a result, there is uncertainty in regards to the search behavior of these \emph{Multi-objective Evolutionary Algorithms (MOEAS)}. This thesis work performs an empirical study about the search of these two MOEAs. The performed analysis develops a series of experiments that compare several DE variants and PSO flight formulas. Thereafter, the manner in which these two MOEAs generate new solutions is evaluated and certain characteristics of these solutions and their relationship to the mechanisms found on a MOEA are identified. These experiments allowed to conclude that MOPSO performs an aggressive search towards promissory regions which might result in stagnation of the search. On the other hand, MODE performs a more passive search taking small steps which on the long run allowed it to continue improving towards the true Pareto front. This knowledge was further used to design two new MOEAs. Results indicate that both algorithms are very competitive with respect to three algorithms (OMOPSO, NSGA-II, and DEMO) representative of the state of the art. 
}

\publication {
Jorge Sebastian Hernández Domínguez and Gregorio Toscano Pulido. \emph{A Comparison on the Search of Particle Swarm Optimization and Differential Evolution on Multi-Objective Optimization}, in IEEE Congress on Evolutionary Computation (CEC 2011), New Orleans, LA, USA, June 2011.
}

\publication {
Jorge Sebastian Hernández Domínguez, Gregorio Toscano Pulido, and Carlos A. Coello Coello, \emph{A Multi-objective Particle Swarm Optimizer Enhanced with a Differential Evolution Scheme}, in International Conference on Artificial Evolution (EA 2011), Angers, France, October 2011.
}

\acknowledgments {
\begin{itemize}
\item { Le agradezco a spiderman y pongan m\'as agradecimientos como los siguientes:}
\item { I also thank the administrative personnel at CINVESTAV-Tamaulipas for their help during my stay}
\item { I thank CONACyT for the provided economic support which allowed me to concentrate in my studies and CINVESTAV-Tamaulipas for the opportunity to pursue graduate studies}
\item { I also acknowledge support from CONACyT through project 105060 ``Uso de técnicas evolutivas híbridas para resolver problemas de optimización multiobjetivo dinámicos y con más de tres objetivos'' under the lead of Dr. Gregorio Toscano Pulido}
\end{itemize}
}

\nomenclature {
\begin{longtable}{ll}
\textbf{BBDE} & Bare Bones Differential Evolution\tabularnewline
\textbf{BBPSO} & Bare Bones Particle Swarm Optimization\tabularnewline
\textbf{DelMiDE} & Delayed Micro Differential Evolution\tabularnewline
\textbf{DTLZ} & Deb-Thiele-Leumman-Zitzler test suite\tabularnewline
\textbf{EA} & Evolutionary Algorithm\tabularnewline
\textbf{EC} & Evolutionary Computation\tabularnewline
\textbf{ES} & Evolutionary Strategy\tabularnewline
\textbf{EP} & Evolutionary Programming\tabularnewline
\textbf{DE} & Differential Evolution\tabularnewline
\textbf{DEMO} & An specific implementation of Multi-objective Differential Evolution\tabularnewline
\textbf{GA} & Genetic Algorithm\tabularnewline
\textbf{GD} & Generational Distance\tabularnewline
\textbf{IFF} & If and only if\tabularnewline
\textbf{IGD} & Inverted Generational Distance\tabularnewline
\textbf{MODE} & Multi-objective Differential Evolution\tabularnewline
\textbf{MOEA} & Multi-objective Evolutionary Algorithm\tabularnewline
\textbf{MOP} & Multi-objective Optimization Problem\tabularnewline
\textbf{MOPEDS} & Multi-objective Particle Swarm Optimization Enhanced with a Differential Evolution Scheme\tabularnewline
\textbf{MOPSO} & Multi-objective Particle Swarm Optimizer\tabularnewline
\textbf{NSGA-II} & Non-dominated Sorting Genetic Algorithm\tabularnewline
\textbf{PF} & Pareto Optimal Front\tabularnewline
\textbf{PS} & Pareto Optimal Set\tabularnewline
\textbf{PSO} & Particle Swarm Optimization\tabularnewline
\textbf{SMPSO} & Speed constrained Multi-objective Particle Swarm Optimization\tabularnewline
\textbf{WRT} & With Respect To\tabularnewline
\textbf{ZDT} & Zitzler-Deb-Thiele test suite\tabularnewline
\tabularnewline
\tabularnewline
\end{longtable}
}

 \floatname{algorithm}{Algorithm}
%\renewcommand{\algorithmicrequire}{\textbf{Entrada:}}
%\renewcommand{\algorithmicensure}{\textbf{Salida:}}

%\@ifundefined{showcaptionsetup}{}{%
 %\PassOptionsToPackage{caption=false}{subfig}}
%\usepackage{subfig}
\makeatother

%\usepackage{babel}
%\addto\shorthandsspanish{\spanishdeactivate{~<>}}

\setcounter{lofdepth}{1} 
\setcounter{lotdepth}{1} 

\begin{document}
%\maketitle
\makeintropages

%\renewcommand{\tablename}{Tabla}

\begin{doublespace}
\chapter{Introduction}

Optimization is the act of obtaining the best result under given circumstances \cite{Rao2009}. In a typical engineering problem, optimization can be seen as the goal of minimizing effort or maximizing benefit. Given these very desirable goals, optimization has been a widely studied research area for several years to date. This popularity has promoted the development of many classical techniques such as the simplex method for linear optimization and several other direct search and gradient based methods for non-linear optimization. Even though many optimization problems have been successfully solved using classical techniques, many others can not be solved with these methods due to the size of the feasible region, their rough surface, or the impossibility to differentiate the objective function. These issues suggest the need for alternative optimization approaches such as evolutionary based algorithms. 

%%%%%%%%%%%%%%%%%%%%%%%
%%%%%%%%%%%%%%%%%%%%%%%
%%%%%%%%%%%%%%%%%%%%%%%
%%%%%%%%%%%%%%%%%%%%%%%
%%%%%%%%%%%%%%%%%%%%%%%

\textbf{\emph{Evolutionary Algorithms (EAs)}} are metaheuristics inspired by Neo-Darwinian evolution theory. Two evolutionary algorithms which have been successfully applied in a wide variety of optimization tasks are: $i$) \emph{Particle Swarm Optimization (PSO)} \cite{Kennedy1995} and $ii$) \emph{Differential Evolution (DE)} \cite{Price1997}. The PSO algorithm is based on the mimicking of flocks in the search for food. PSO relies on a population of particles, referred as the \emph{swarm}, where each particle represents a solution to the problem at hand and flies on a $n$-dimensional space. DE on the other hand, is based on the premise that the population itself is a convenient source for perturbation. Therefore, on DE, the aim is to move solutions based on the variance of the population.  Both PSO and DE have shown excellent results on single objective optimization and their success has promoted their use on \emph{Multi-objective Optimization Problems (MOPs)}.

The first \emph{Multi-objective Particle Swarm Optimizer (MOPSO)} was proposed by Moore and Chapman \cite{Moore99}. Since then, several MOPSOs \cite{Reyes06,Toscano05,Branke06,Coello04} have emerged which solve non-trivial MOPs using a very low number of function evaluations. For the case of differential evolution, Abbass \emph{et al.} \cite{Abbass01} reported the first proposal of a \emph{Multi-objective Differential Evolution (MODE)}. As in the case of MOPSO, from this point several MODE algorithms have been proposed in the specialized literature \cite{Robic05,Santana05,Xue03}.


\section{Problem statement} \label{}

Even when several MOPSO and MODE proposals are available in the state of the art, very few researchers have studied the mechanisms that guide the search on these two \emph{Multi-objective Evolutionary Algorithms} (MOEAs). In specific, some previous research have developed theoretical analysis of these metaheuristics \cite{Vandenbergh2006,Clerc2002} and even when the results obtained from these works have improved both PSO and DE, most of them are for single-objective optimization. Moreover, results rely on a set of simplifications made to the metaheuristic at hand and/or a mathematical model that describes the movement of solutions. In addition, such analyses focused on one metaheuristic only (either PSO or DE). This prevents these works from comparing the search behavior of one metaheuristic with another such that knowledge in regards to what algorithm would work best under given circumstances is obtained. 

It seems reasonable to think that using an empirical approach to visualize the behavior of MOPSO and MODE might allow for a better understanding of these two MOEAs. In this sense, studying the manner in which these two metaheuristics generate new solutions and finding the relationship between characteristics of these solutions with the mechanisms found in each MOEA might allow to obtain knowledge not found previously. This knowledge may permit to better describe the behavior of an evolutionary technique and, as a result, to take full advantage of the search characteristics of a MOEA. To this end, it seems wise to compare and contrast the search performed by MOPSO and MODE on a series of multi-objective optimization problems with the aim to identify the mechanisms that allow to solve MOPs at a low number of function evaluations. The challenge of such a study relies on finding the mechanisms that improve the search behavior of these two algorithms.

%This thesis work performs a series of experiments on the metaheuristics particle swarm optimization and differential evolution on multi-objective optimization problems. The performed experiments attempt to obtain further information about the search performed by these two metaheuristics. To this end, the first experiment evaluates a series of flight formulas for particle swarm optimization. Then, a similar experiment is performed with regards to several differential evolution variants found in the specialized literature. Thereafter, an online convergence experiment is presented in order to observe the convergence behavior of MOPSO and MODE through time. Moreover, the distribution of the generated points (in decision and objective space) by these two metaheuristics is also addressed. In addition, the distance traveled by particles of MOPSO and solutions of MODE allowed us to obtain further information on the behavior of these two MOEAs. Finally, once enough information was obtained, a PSO-DE hybrid multi-objective evolutionary algorithm is proposed which attempts to use the most beneficial mechanisms from MOPSO and MODE. The proposal is further evaluated with a series of experiments that show its competitiveness with other state of the art MOEAs. \\

\section{Motivation}

The motivation for this work relies on the lack of detailed knowledge about the search behavior of particle swarm optimization and differential evolution on multi-objective optimization. The ability to identify the mechanisms that impact on the search of these two MOEAs will allow for the design of new evolutionary approaches that use a reduced number of objective function evaluations to solve a selection of MOPs. 

\section{Hypothesis}

It is possible to understand the behavior of the evolutionary algorithms \emph{particle swarm optimization} and \emph{differential evolution} on \emph{multi-objective optimization} problems  by means of identifying general characteristics and mechanisms that can be detrimental or beneficial to the search. The identification of those mechanisms and the understanding of their promoted behavior is beneficial for the development of new MOEAs that are better suited for MOPs. 

\section{Objectives}

The main objective of this thesis work can be defined as follows: \emph{``To contribute to the state of the art by identifying and understanding the mechanisms that promote good performance on multi-objective particle swarm optimization and multi-objective differential evolution and that allow a multi-objective evolutionary algorithm to reduce the number of objective function evaluations needed to solve an arbitrary MOP''}. 

This main goal has been divided in the following specific objectives: 
\begin{itemize}\setlength{\itemsep}{-1mm}
	\item To evaluate different strategies to generate solutions in DE and to move particles in PSO on multi-objective problems.
	\item To perform a series of experiments that allow to better understand the behavior of PSO and DE on multi-objective problems.
	\item To identify the mechanisms that impact (either enhance it or deter it) in the search of the metaheuristics particle swarm optimization and differential evolution when attacking multi-objective optimization problems. 
	\item To contribute to the state of the art with at least one multi-objective evolutionary algorithm that utilizes knowledge derived from this work to enhance the search process on diversity and convergence. 
	\item To validate the performed experiments using performance measures and test problems taken from the specialized literature.
\end{itemize}


\section{Document outline}

The rest of this document is organized as follows: Chapter~\ref{chapter2} presents basic concepts and background in the field of optimization. Then, Chapter~\ref{ch:PSODE} introduces particle swarm optimization and differential evolution which are the two metaheuristics on which this thesis work focusses. In order to introduce these two metaheuristics, EAs is general are also described in this chapter. Afterwards, Chapter~\ref{experiments} presents a series of experiments that were developed and that allowed to obtain further information about the search performed by PSO and DE in multi-objective optimization. This knowledge was used to develop two new MOEAs which are presented and evaluted in Chapter~\ref{proposals}. Finally, Chapter~\ref{conclusion} concludes this thesis work. 





%\include{chapter2}

%\include{chapter3}

%\include{chapter4}

%\include{chapter5}

%\include{chapter6}

%\include{chapter7}

\end{doublespace}

\appendix

\begin{doublespace}

%\include{appendixA}

%\include{appendixB}

%\bibliographystyle{apalike}
\bibliography{EMOO,protocolo}
\end{doublespace}

\end{document}
